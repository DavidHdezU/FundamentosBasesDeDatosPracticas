\documentclass{article}
\usepackage[utf8]{inputenc}

% Símbolos
\usepackage{recycle}
\usepackage{amsmath}
\usepackage{amsfonts}
\usepackage{amssymb}
\usepackage{float}
\restylefloat{table}

% Short labels
\usepackage[shortlabels]{enumitem}

% Multicol
\usepackage{multicol}

% Figuras
\usepackage{mathrsfs}
\usepackage{amsmath}
\usepackage{caption}
\usepackage[font=footnotesize]{caption}
\usepackage{graphicx}
\graphicspath{{img/}}

% Colores
\usepackage{xcolor}

% Márgenes
\usepackage{geometry}
\geometry{showframe=false, headheight=1cm}
\geometry{margin=30mm, bottom=30mm}

% Title size
\usepackage{titlesec}
\titleformat{\section}
{\normalfont\large\bfseries}{\thesection}{1em}{}

% Header & Footer
\usepackage{fancyhdr}
\renewcommand{\footrulewidth}{0.5pt}
\pagestyle{fancy}
\fancyhf{}
\fancyhead[R]{\footnotesize Practica 04: Modelo Entidad-Relación y Extendido}
\fancyhead[L]{\footnotesize UNAM, Facultad de Ciencias}
\fancyfoot[L]{\footnotesize Fundamentos de Bases de Datos 2022-2}
\fancyfoot[R]{\footnotesize \thepage}

% Custom Commands
\newcommand{\dollar}{\mbox{\textdollar}}

% Import files
\usepackage{subfiles}

\begin{document}

\section*{Restricciones del modelo}

Una de los primeros cambios a resaltar respecto al modelo planteado con anterioridad es que ahora empezamos a manejar 
especialización entre entidades. Como podemos observar principalmente en la especialización total de la entidad \textbf{Persona}, 
en la cuál definimos todos los atributos comunes entre, precisamente, las ``personas'' involucradas en el planteamiento del
problema. Vemos que dichos atributos son: nombre completo, dirección y CURP, donde éste último lo empleamos como llave.

Otro punto a tratar es la especialización parcial por la que obtamos para definir tanto a la entidad \textbf{Cliente Frecuente} 
como a \textbf{Veterinario}, que se derivan de \textbf{Cliente} y \textbf{Supervisor} respectivamente. Si bien éstas dos entidades 
son un tipo de sus entidades superiores, éstas no necesariamente pertenecen a sus entidades de nivel más bajo. El ejemplo más claro 
de ésto es que un Cliente no siempre es Cliente Frecuente.

Pasando ahora a los tipos de atributos empleados para el nuevo modelo, vemos por ejemplo que ``horario'' de Supervisor es un atributo 
compuesto, mientras que para Estética es multivaluado. Ésto debido al mismo planteamiento del problema, donde nos piden específicamnte 
guardar las horas de entrada y salida de éste tipo de trabajador.
Así mismo vemos que el atributo ``código'' de la entidad \textbf{Emergencia} (la cuál es una especialización de Consulta) es multivaluado,
ésto para manejar los diferentes tipos de códigos.\\ 
Luego, hacemos uso de un atributo calculado denomidado ``precio'' en la entidad \textbf{Producto} que irá cambiando según el tipo de 
producto.

Ahora, para las relaciones:
\begin{itemize}
    \item Trabajar. Varios a uno entre Trabajador y Estética. Aún no definimos cardinalidad puesto que no se nos está especificando 
    cuántos trabajadores tendrá cada Estética.
    \item Tener. Varios a uno entre Mascota y Cliente. Aquí queda claro que un mismo Cliente puede tener 
    ninguno o varios(as) Mascotas.
    \item Vender. Varios a uno Producto y Estética. Mismo caso que en la relación Tener.
    \item Adiministrar. Varios a uno entre Consultorio y Estética. Puesto que nos dicen que una misma Estética puede tener hassta 4 Colsutorio(s),
    nos decidimos por la cardinalidad mostrada en el diagrama (0...4).
    \item Requerir. Varios a uno entre Consulta y Mascota. No hemos definido una cardinalidad en específico para ésta relación puesto que 
    nada más nos están aclarando que una misma Mascota puede requerir varias consultas.
    \item Poseer. Varios a uno entre Pago y Cliente. Puesto que un cliente puede pagar de diferentes maneras, nos decidimos por definir así 
    ésta relación.
\end{itemize}


Por último, tenemos varios puntos de mejora, como lo son:
\begin{itemize}
    \item Relacionar Consulta con las demás entidades que deberían participar en una ``consulta'', como lo son Veterinario y Consultorio.
    \item Terminar de definir los atributos de Consultorio.
    \item Decidirnos si nos conviene más relacionar las especializaciones de Trabajador a Estética dependiendo de sus papeles.
    \item Entender como un Cliente puede pagar por los bienes o servicios obtenidos en la Estética.
    \item Ver la manera en que van a interactuar Cliente Frecuente y Estética para las felicitaciones.
    \item Llevar el control de inventario y los ingresos de la Estética.
    \item Y definir alguna entidad de los recibos para los clientes.
\end{itemize}

Creemos que todavía tenemos mucho espacio para mejorar el diseño de la solución pero estamos de acuerdo que ésta mejoría se irá presentando 
conforme nos vayan dando más información sobre los requerimientos.


\end{document}
